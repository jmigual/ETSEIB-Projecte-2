\documentclass[a4paper]{article}

\usepackage[margin=2cm]{geometry}
\usepackage{fontspec}
\usepackage[catalan]{babel}
\usepackage{graphicx}
\usepackage[
	backend=biber,
	citestyle=numeric  
]{biblatex}

\setlength{\parindent}{0pt}
\setlength{\parskip}{0.5em}
\def\arraystretch{1.5}

\nocite{*}
\bibliography{bibliografia.bib}

\begin{document}

\begin{titlepage}
	\centering
	\vspace{1cm}
	\includegraphics[width=0.25\textwidth]{images/etseib}
	\par\vspace{1cm}
	\textsc{ \LARGE Escola Tècnica Superior d'Enginyeria \\[1em] 
		Industrial de Barcelona}
	\par\vspace{2cm}
	\textbf{\LARGE Projecte II}
	\par\vspace{2cm}
	{\Huge Construcció d'una orquestra amb Raspberry Pi}
	\vfill
	\begin{flushright}
		\large
		Arnau Canyadell i Miquel \par
		Joan Marcè i Igual \par
	\end{flushright}
\end{titlepage}

\tableofcontents

\newpage

\section{Introducció}

\subsection{Motivacions}

Donat que els dos creadors d'aquest projecte som músics i que tots dos hem participat en diverses orquestres el projecte de construir una orquestra mitjançant la tecnologia i \emph{Internet of Things} ens va semblar molt interessant i que podria integrar-se bé amb els nostres coneixements.

\subsection{Projectes similars}

Buscant si hi ha altres projectes semblants al que es vol realitzar s'ha trobat que Google l'any 2011 va crear el projecte \emph{YouTube Symphony Orchestra 2011}\cite{YoutubeSymphoni} que tenia com a objectiu poder fer una orquestra a nivell mundial. Per fer-ho tots els músics es connectaven a internet i després tocaven tots junts i tot seguit totes les pistes d'àudio que aquests generaven s'ajuntaven formant una orquestra.

A part també hi ha un projecte anomenat \emph{Orchestra! a Distributed Platform for Virtual Musical Groups and Music Distance Learning over the Internet in Java \texttrademark Technology} \cite{Orchestra}. Que permet a diversos grups de música assajar o aprendre conjuntament a través d'internet tot i que no es trobin junts físicament.

\section{Objectius}

\section{Desenvolupament del treball}

\section{Resultats}

\printbibliography

\end{document}