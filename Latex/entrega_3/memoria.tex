\documentclass[a4paper]{article}

\usepackage[margin=2cm]{geometry}
\usepackage{fontspec}
\usepackage[catalan]{babel}
\usepackage{graphicx}
\usepackage{url}

\title{Construcció d'una orquestra amb Raspberry Pi}
\author{Arnau Canyadell Miquel \and Joan Marcè Igual}
\date{Primavera de 2017}

\begin{document}

\begin{titlepage}
	\centering
	\vspace{1cm}
	\includegraphics[width=0.25\textwidth]{images/etseib}
	\par\vspace{1cm}
	\textsc{ \LARGE Escola Tècnica Superior d'Enginyeria \\[1em] 
		Industrial de Barcelona}
	\par\vspace{2cm}
	\textbf{\Huge Projecte II}
	\par\vspace{2cm}
	{\LARGE Construcció d'una orquestra amb Raspberry Pi}
	\vfill
	\begin{flushright}
		\large
		Arnau Canyadell i Miquel \par
		Joan Marcè i Igual \par
	\end{flushright}
\end{titlepage}

\tableofcontents

\newpage

\section{Introducció}

\subsection{Motivacions}

Donat que els dos creadors d'aquest projecte som músics i que tots dos hem participat en diverses orquestres el projecte de construir una orquestra mitjançant la tecnologia i \emph{Internet of Things} ens va semblar molt interessant i que podria integrar-se bé amb els nostres coneixements.

\subsection{Altres projectes}

Buscant si hi ha altres projectes semblants al que es vol realitzar s'ha trobat que Google l'any 2011 va crear el projecte \emph{YouTube Symphony Orchestra 2011}

\section{Objectius}

\section{Desenvolupament del treball}
El treball s'ha dut a terme durant un el quadrimestre de primavera, des de mitjans de febrer fins a finals de maig. S'ha estructurat la feina en 14 sessions, fent defenses parcials del projecte a les sessions 5 i 9, i una defensa final en l'última.

\subsection{Eines de treball}
Degut a ala naturalesa de desenvolupament de programari informàtic del projecte i de la necessitat de mantenir un historial de versions i un ordre del codi, s'ha treballat amb un \textbf{repositori públic de GitHub} (\url{https://github.com/jmigual/projecte2/}).

Per a l'assignació de tasques dintre del grup s'ha utilitzat l'eina \textbf{Trello}, que permet crear tasques amb data límit, assignar-les a una persona i marcar tasques com a realitzades o endarrerides, entre altres coses.

Python

\LaTeX

\subsection{Evolució del projecte}
En el repositori hi ha un document, \texttt{documentacio.md}, on es detalla l'historial de feina feta en cada sessió (començant en la sessió 3).

\subsubsection{Inici (sessions 1 a 3)}
En les primeres sessions es va decidir el tema i es van establir els objectius del projecte. També es van definir les eines de treball que s'utilitzarien durant el projecte: es va crear el repositori a GitHub i es va crear el Trello. D'altra banda, es va buscar documentació sobre el FluidSynth i el Polyphone i es va analitzar el codi del director que ja estava implementat. Finalment, es va aprendre a configurar la raspberry pi i es va configurar la raspberry que s'ha utilitzat per fer les proves al llarg de tot el desenvolupament.

\subsubsection{Programació del director i els músics (sessions 4 a 9)}

\subsubsection{Implementació del bot de Telegram i reproducció d'arxius MIDI (sessions 10 a 14)}

\section{Resultats}

\section{Bibliografia}


\end{document}