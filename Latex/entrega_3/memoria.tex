\documentclass[a4paper]{paper}

\usepackage[catalan]{babel}
\usepackage[
backend=biber,
citestyle=numeric  
]{biblatex}
\usepackage{fontspec}
\usepackage[margin=2cm]{geometry}
\usepackage{graphicx}
\usepackage{titlesec}
\usepackage{url}
\usepackage{listings}

\setlength{\parindent}{0pt}
\setlength{\parskip}{0.5em}
\def\arraystretch{1.5}

\nocite{*}
\bibliography{bibliografia.bib}

\let\oldsection\section
\renewcommand\section{\clearpage\oldsection}

\begin{document}

\begin{titlepage}
	\centering
	\vspace{1cm}
	\includegraphics[width=0.25\textwidth]{images/etseib}
	\par\vspace{1cm}
	\textsc{ \LARGE Escola Tècnica Superior d'Enginyeria \\[1em] 
		Industrial de Barcelona}
	\par\vspace{2cm}
	\textbf{\LARGE Projecte II}
	\par\vspace{2cm}
	{\Huge Construcció d'una orquestra amb Raspberry Pi}
	\vfill
	\begin{flushright}
		\large
		Arnau Canyadell i Miquel \par
		Joan Marcè i Igual \par
	\end{flushright}
\end{titlepage}

\tableofcontents

\newpage

\section{Introducció}

\subsection{Motivacions}

Donat que els dos creadors d'aquest projecte som músics i que tots dos hem participat en diverses orquestres el projecte de construir una orquestra mitjançant la tecnologia i \emph{Internet of Things} ens va semblar molt interessant i que podria integrar-se bé amb els nostres coneixements.

\subsection{Projectes similars}

Buscant si hi ha altres projectes semblants al que es vol realitzar s'ha trobat que Google l'any 2011 va crear el projecte \emph{YouTube Symphony Orchestra 2011}\cite{YoutubeSymphoni} que tenia com a objectiu poder fer una orquestra a nivell mundial. Per fer-ho tots els músics es connectaven a internet i després tocaven tots junts i tot seguit totes les pistes d'àudio que aquests generaven s'ajuntaven formant una orquestra.

A part també hi ha un projecte anomenat \emph{Orchestra! a Distributed Platform for Virtual Musical Groups and Music Distance Learning over the Internet in Java \texttrademark Technology} \cite{Orchestra}. Que permet a diversos grups de música assajar o aprendre conjuntament a través d'internet tot i que no es trobin junts físicament.

\section{Objectius}
A l'inici del projecte es van establir uns objectius a complir en el projecte. Aquests es van dividir en dos grups. D'una banda, els objectius a curt termini, objectius que estaven ben definits i s'havien de fer sí o sí abans que s'acabés el projecte; i d'altra banda hi havia els objectius o idees a llarg termini, que no estaven tan ben definits i que s'intentarien completar si hi havia temps.

\subsection{Objectius a curt termini}
\begin{itemize}
	\item Aconseguir rebre la música d'un ordinador
	\item Enviar la música a una Raspberry Pi
	\item Reproduir la música en una Raspberry Pi
\end{itemize}

\subsection{Objectius a llarg termini}
\begin{itemize}
	\item Programar un bot de Telegram que reprodueixi amb l'orquestra de Raspberry Pis la música que se li enviï.
	\item Enviar música del mòbil a l'ordinador.
\end{itemize}

\section{Desenvolupament del treball}
El treball s'ha dut a terme durant un el quadrimestre de primavera, des de mitjans de febrer fins a finals de maig. S'ha estructurat la feina en 14 sessions, fent defenses parcials del projecte a les sessions 5 i 9, i una defensa final en l'última.

\subsection{Eines de treball}
Degut a ala naturalesa de desenvolupament de programari informàtic del projecte i de la necessitat de mantenir un historial de versions i un ordre del codi, s'ha treballat amb un \textbf{repositori públic de GitHub} (\url{https://github.com/jmigual/projecte2/}).

Per a l'assignació de tasques dintre del grup s'ha utilitzat l'eina Trello\cite{trello}, que permet crear tasques amb data límit, assignar-les a una persona i marcar tasques com a realitzades o endarrerides, entre altres coses.

Per al desenvolupament del programari s'ha utilitzat majoritàriament el llenguatge Python\cite{python}, amb el qual està fet tot el codi de l'orquestra, inclosos director, músics i bot. S'ha utilitzat l'entorn de Pycharm\cite{pycharm} per a escriure codi, executar-lo i depurar-lo en aquest llenguatge. Per a la configuració de les Raspberry Pis s'ha utilitzat shell script, i també s'ha treballat molt amb comandes de git.

Finalment, totes les presentacions del projecte i la memòria s'han fet amb \LaTeX.

\subsection{Evolució del projecte}
En el repositori hi ha un document, \texttt{Documentacio.md}, on es detalla l'historial de feina feta en cada sessió (començant en la sessió 3).

\subsubsection{Inici (sessions 1 a 3)}
En les primeres sessions es va decidir el tema i es van establir els objectius del projecte. També es van definir les eines de treball que s'utilitzarien durant el projecte: es va crear el repositori a GitHub i es va crear el Trello. D'altra banda, es va buscar documentació sobre el FluidSynth i el Polyphone i es va analitzar el codi del director que ja estava implementat. Finalment, es va aprendre a configurar la Raspberry Pi i es va configurar la raspberry que s'ha utilitzat per fer les proves al llarg de tot el desenvolupament.

\subsubsection{Programació del director i els músics (sessions 4 a 9)}
Un cop endinsats en el projecte, es va començar a desenvolupar el programari. Es va programar, primer de tot, una versió del músic, que rebia la música del director i la reproduïa. En aquest punt es va trobar un problema amb la configuració de la sortida d'àudio de la Raspberry Pi: la sortida per defecte és el minijack, i no se sabia com canviar-la a la targeta USB. Per tal de solucionar-ho es va provar de canviar diverses configuracions, i finalment es va optar per instal·lar pixel a l'ordinador monoplaca i canviar la configuració manualment des de l'entorn gràfic. Així es va aconseguir solucionar el problema però el canvi implicava connectar-se manualment a l'entorn gràfic cada cop que s'iniciava la Raspberry Pi.

Un cop tot funcionava es modificar el director. Es van crear les classes Director i Nota per separar, ordenar i simplificar el codi. En la segona defensa del projecte, a la sessió 9, es va poder dur a terme la primera prova de l'orquestra amb diversos músics. Al final d'aquesta fase del treball ja s'havien complert els objectius a curt termini.

\subsubsection{Implementació del bot de Telegram i reproducció d'arxius MIDI (sessions 10 a 14)}
Finalment es va trobar una manera de canviar la sortida d'àudio de les Raspberry Pis i es va crear un script per canviar-lo de forma simplificada. Després d'això es va treballar en l'objectiu de crear un bot de Telegram que reproduís la música que se li envia, i així complir els dos objectius a llarg termini. Per a dur a terme aquest objectius es va para\l.lelitzar la feina en la creació del bot i en la reproducció d'arxius MIDI.

\section{Funcionament de l'orquestra}

\subsection{Comunicació entre director i intèrprets}


\subsection{Bot de Telegram}


\subsection{Reproducció d'arxius MIDI}
Per a la reproducció dels arxius MIDI s'ha usat la llibreria Mido\cite{mido}. Que permet la reproducció en temps real d'aquest tipus d'arxius. Aquesta llibreria s'ha modificat lleugerament per poder obtenir tots els canals del MIDI mentre es reprodueix de manera que pot ser configurat amb el sistema de múltiples Raspberry Pi per reproduir que s'usa en aquest projecte.

Així doncs, un cop la connexió amb el bot de Telegram està finalitzada és tant senzill com passar el camí fins a l'arxiu que es vulgui reproduir i el nou director s'encarrega de demanar-li a la llibreria la informació necessària.

El codi per reproduir queda així de la següent manera:
\begin{lstlisting}
# file_path conté el camí fins a l'arxiu
mid = DirectorMidiFile(file_path)
for missatge, track in mid.play_tracks():
	msg_in, msg_out = descodifica(missatge)
	json_string = json.dumps({
		"in": msg_in,
		"out": msg_out
	})
	socket.sendto(json_string.encode(), multicast_group)
\end{lstlisting}


\section{Resultats}



\printbibliography

\end{document}